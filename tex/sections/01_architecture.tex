\section{제1장: 듀얼 인코더 & Titans 통합 아키텍처 (Embodied Neuro-AI Architecture)}

\subsection{1.1 개요: 신체화된 신경 인공지능 (Embodied Neuro-AI)}
본 제안서는 인간의 인지 과정이 뇌 내부 연산뿐만 아니라 신체(Body)와 환경(Environment)의 상호작용 속에서 발현된다는 \textbf{"Embodied AI"} 철학을 따름. 시각/청각 등 외부 감각(External Sense)과 뇌파/고유감각 등 내부 신호(Internal Signal)를 통합하여, 인간 수준의 적응성과 강건성을 갖춘 \textbf{"DIVER-Neuro 파운데이션 모델"}을 제안함.

\begin{figure}[h]
\centering
\includegraphics[width=1.0\textwidth]{dual_encoder_titans_architecture.png}
\caption{Dual Encoder Titans Architecture}
\end{figure}

\subsection{1.2 듀얼 인코더 시스템 (Dual Encoders)}
서로 다른 물리적 특성을 가진 두 가지 데이터 스트림 처리를 위해 특화된 \textbf{Dual Encoder} 구조를 채택함.

\subsubsection{1.2.1 Sensory-Motor Encoder (감각-운동 인코더): \textbf{OmniField}}
에이전트가 환경과 상호작용하는 데 필요한 외부 정보를 처리함.
\begin{itemize}
\item   \textbf{OmniField (Neural Fields)}: David Keetae Park 등\cite{valencia2025omnifield}이 제안한 \textbf{OmniField} 아키텍처를 도입함. 이는 희소(Sparse)하고 비정형적인(Irregular) 멀티모달 데이터를 강건한 연속 함수(Continuous Function)로 인코딩하여, 노이즈가 심한 생체/환경 데이터 처리에 최적화됨.
\item   \textbf{Proprioception Integration}: 기존 ViT의 고정된 그리드 처리 한계를 넘어, \textbf{OmniField}의 유연한 좌표 기반(Coordinate-based) 신경망을 통해 관절 각도 및 촉각 정보를 시각 정보와 매끄럽게 통합함.
\end{itemize}

\subsubsection{1.2.2 Brain Spatiotemporal Encoder (뇌 시공간 인코더)}
뇌 활동(EEG/iEEG)의 복잡한 시공간적 동역학(Spatiotemporal Dynamics)을 모델링함.
\begin{itemize}
\item   \textbf{Liquid Time-Constant (LTC) Networks}: 불규칙하고 연속적인 뇌 신호 처리를 위해, 입력에 따라 유동적으로 변하는 \textbf{Time-constant}를 가진 미분방정식 기반 \textbf{LTC (Neural ODE)}를 도입함[2].
\item   \textbf{장점}: 노이즈가 많은 생체 신호 처리 시 기존 RNN/LSTM 대비 월등한 강건성을 보이며, 인과관계(Causality) 추론에 유리함.
\end{itemize}

\subsection{1.3 Titans Memory: 통합과 선택적 주의 (The Integrator)}
이종(Heterogeneous) 표상은 \textbf{Titans Memory Core}에서 통합됨.
\begin{itemize}
\item   \textbf{Neural Memory Module (MIRAS Framework)}: Google의 최신 \textbf{MIRAS (Memory as Context)} 프레임워크\cite{behrens2025titans}에 기반하여, 단순 저장이 아닌 \textbf{"Surprise Metric (놀라움 척도)"}과 \textbf{"Momentum (관성)"}을 이용해 능동적으로 중요 기억을 선별함.
\item   \textbf{Global Neural Workspace (GNW)}: Dehaene 등[5]의 GNW 이론에 따라, Titans는 감각/뇌 정보가 경쟁·통합되는 "의식적 작업 공간" 역할을 수행함.
\item   \textbf{Graph-Informed Reasoning}: Moontae Lee(LG AI Research) 심층 추론 방법론\cite{lee2024strategic}을 적용, Titans 메모리에 저장된 에피소드들을 그래프 구조로 연결하여 단순 회상(Recall)을 넘어선 \textbf{전략적 추론(Strategic Reasoning)}을 수행함.
\item   \textbf{Tubular Manifold Alignment}: 통합 과정에서 모델은 Bertram 등[4]의 \textbf{"Tubularity(관형 구조)"}를 유지하도록 학습되어 강건성을 확보함.
\end{itemize}

\subsection{1.4 결론 및 차별점}
본 아키텍처는 단순 멀티모달 결합(Early Fusion)을 넘어, \textbf{LTC 기반 생물학적 동역학 모델링}과 \textbf{Titans 기반 능동적 메모리 관리}를 결합한 세계 최초 시도임. 이를 통해 IITP RFP가 요구하는 "인간 수준 인지 능력을 갖춘 뇌 내재화 모델"을 실현함.
