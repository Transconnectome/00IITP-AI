\section{제1장: 듀얼 인코더 & Titans 통합 아키텍처 (Embodied Neuro-AI Architecture)}

\subsection{1.1 개요: 신체화된 신경 인공지능 (Embodied Neuro-AI)}
본 제안서는 인간의 인지 과정이 뇌 내부 연산뿐만 아니라 신체(Body)와 환경(Environment)의 상호작용 속에서 발현된다는 \textbf{"Embodied AI"} 철학을 따름. 시각/청각 등 외부 감각(External Sense)과 뇌파/고유감각 등 내부 신호(Internal Signal)를 통합하여, 인간 수준의 적응성과 강건성을 갖춘 \textbf{"DIVER-Neuro 파운데이션 모델"}을 제안함.

\begin{figure}[h]
\centering
\includegraphics[width=1.0\textwidth]{graphics/dual_encoder_titans_architecture.png}
\caption{Dual Encoder Titans Architecture}
\end{figure}

\subsection{1.2 듀얼 인코더 시스템 (Dual Encoders)}
서로 다른 물리적 특성을 가진 두 가지 데이터 스트림 처리를 위해 특화된 \textbf{Dual Encoder} 구조를 채택함.

\subsubsection{1.2.1 Sensory-Motor Encoder (감각-운동 인코더)}
에이전트가 환경과 상호작용하는 데 필요한 외부 정보를 처리함.
\begin{itemize}
\item   \textbf{Visual & Text}: ViT 및 LLM 백본을 활용, 시각적 객체와 언어적 의미를 고차원 잠재 공간으로 매핑함.
\item   \textbf{Proprioception & Tactile (신규)}: 기존 멀티모달 모델 한계를 극복하고자 \textbf{고유감각(Proprioception)}과 촉각(Tactile) 정보를 명시적으로 포함함[RFP-013]. 관절 각도(Joint Angles) 및 피부 압력 분포 등을 처리하여 에이전트의 "신체 도식(Body Schema)"을 형성함.
\end{itemize}

\subsubsection{1.2.2 Brain Spatiotemporal Encoder (뇌 시공간 인코더)}
뇌 활동(EEG/iEEG)의 복잡한 시공간적 동역학(Spatiotemporal Dynamics)을 모델링함.
\begin{itemize}
\item   \textbf{Liquid Time-Constant (LTC) Networks}: 불규칙하고 연속적인 뇌 신호 처리를 위해, 입력에 따라 유동적으로 변하는 \textbf{Time-constant}를 가진 미분방정식 기반 \textbf{LTC (Neural ODE)}를 도입함[2].
\item   \textbf{장점}: 노이즈가 많은 생체 신호 처리 시 기존 RNN/LSTM 대비 월등한 강건성을 보이며, 인과관계(Causality) 추론에 유리함.
\end{itemize}

\subsection{1.3 Titans Memory: 통합과 선택적 주의 (The Integrator)}
이종(Heterogeneous) 표상은 \textbf{Titans Memory Core}에서 통합됨.
\begin{itemize}
\item   \textbf{Neural Memory Module}: Titans 아키텍처[1]는 Test-time에 과거 맥락을 효율적으로 기억(Memorize)함. 이는 단순 저장이 아닌, 현재 목표(Goal) 관련 정보만 선별하는 \textbf{"Memory Surprise"} 메트릭에 기반함.
\item   \textbf{Global Neural Workspace (GNW)}: Dehaene 등[5]의 GNW 이론에 따라, Titans는 감각/뇌 정보가 경쟁·통합되는 "의식적 작업 공간" 역할을 수행함. \textbf{Top-down Selective Attention}이 작동하여 무의미한 노이즈를 억제하고 행동에 필요한 핵심 정보만 장기 기억으로 전이함.
\item   \textbf{Tubular Manifold Alignment}: 통합 과정에서 모델은 Bertram 등[4]의 \textbf{"Tubularity(관형 구조)"}를 따르도록 학습됨. 잠재 궤적(Latent Trajectory)이 엉키지 않고(Untangled) 매끄러운 튜브 형태를 유지하여, 환경 변화에도 안정적 성능을 보장함.
\end{itemize}

\subsection{1.4 결론 및 차별점}
본 아키텍처는 단순 멀티모달 결합(Early Fusion)을 넘어, \textbf{LTC 기반 생물학적 동역학 모델링}과 \textbf{Titans 기반 능동적 메모리 관리}를 결합한 세계 최초 시도임. 이를 통해 IITP RFP가 요구하는 "인간 수준 인지 능력을 갖춘 뇌 내재화 모델"을 실현함.
