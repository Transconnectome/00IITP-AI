\section{듀얼 인코더 & Titans 통합 아키텍처 (Embodied Neuro-AI Architecture)}

\subsection{개요: 신체화된 신경 인공지능 (Embodied Neuro-AI)}
본 제안서는 인간의 인지 과정이 뇌 내부 연산뿐만 아니라 신체(Body)와 환경(Environment)의 상호작용 속에서 발현된다는 \textbf{"Embodied AI"} 철학을 따름. 시각/청각 등 외부 감각(External Sense)과 뇌파/고유감각 등 내부 신호(Internal Signal)를 통합하여, 인간 수준의 적응성과 강건성을 갖춘 \textbf{"DIVER-Neuro 파운데이션 모델"}을 제안함.

\begin{figure}[h]
\centering
\includegraphics[width=1.0\textwidth]{dual_encoder_titans_architecture.png}
\caption{Dual Encoder Titans Architecture}
\end{figure}

\subsection{듀얼 인코더 시스템 (Dual Encoders)}
서로 다른 물리적 특성을 가진 두 가지 데이터 스트림 처리를 위해 특화된 \textbf{Dual Encoder} 구조를 채택함.

\subsubsection{Sensory-Motor & Interoceptive Encoder (OmniField)}
\textbf{OmniField} \cite{valencia2025omnifield}는 시각/청각뿐만 아니라 \textbf{내수용 감각(Interoception)} 신호를 통합하는 4D Neural Field입니다.
*   \textbf{External}: Camera(Ego-centric), Mic(Audio)의 비정형 데이터를 처리함.
*   \textbf{Internal (Allostasis)}: 위장 팽창(Gastric Distension)과 같은 생체 신호를 연속적인 \textbf{내부 상태 변수(Internal State Variable)}로 인코딩합니다.
    *   \textbf{Biological Grounding}: 김성연 교수팀\cite{kim2020neural}의 연구에 기반하여, \textbf{NTS $\rightarrow$ PVH $\rightarrow$ aIC} 신경 회로를 모사합니다. 위장 팽창 신호를 단순 자극이 아닌 '적분형(Slow Integrator)' 정보로 처리하여 행동 모드(Exploration vs. Exploitation) 전환의 기준점으로 삼습니다.
*   \textbf{Latent Sampler}: 연속적인 OmniField의 출력을 Titans가 처리 가능한 토큰($z_t$)으로 변환하기 위해, \textbf{좌표 기반 샘플링(Coordinate Sampling)} 층을 둠. 이 과정에서 Han et al.\cite{han2024simulated}의 \textbf{Simulated Annealing} 기법을 초기화 단계에 적용하여, 최적의 특징점(Feature Points)을 포착하고 Local Minima를 방지함.

\subsubsection{Brain Spatiotemporal Encoder (뇌 시공간 인코더)}
뇌 활동(EEG/iEEG)의 복잡한 시공간적 동역학(Spatiotemporal Dynamics)을 모델링하기 위해, \textbf{'Hybrid Liquid-SSM Architecture'}를 제안함.
\begin{itemize}
\item   \textbf{Local Micro-Dynamics (LTC)}: 미세한 뉴런 단위의 \textbf{불규칙한 스파이킹(Irregular Spiking)}과 \textbf{인과성(Causality)}을 보존하기 위해, 이산화 오차가 없는 연속 시간 모델인 \textbf{Liquid Time-Constant (LTC) Network}를 적용함[2].
\item   \textbf{Global Macro-Dynamics (SSM)}: 뇌 전체의 장기적 상태 변화(Long-range Dependencies)는 선형 복잡도($O(N)$)를 가진 \textbf{State Space Model (Mamba)} 메커니즘으로 처리하여 BrainMamba\cite{brainmamba2024} 수준의 연산 효율성을 극대화함.
\end{itemize}

\subsection{Titans Memory: 통합과 선택적 주의 (The Integrator)}
이종(Heterogeneous) 표상은 \textbf{Titans Memory Core}에서 통합됨.
\begin{itemize}
\item   \textbf{Neural Memory Module (MIRAS Framework)}: Google의 최신 \textbf{MIRAS (Memory as Context)} 프레임워크\cite{behrens2025titans}에 기반하여, 단순 저장이 아닌 \textbf{"Surprise Metric (놀라움 척도)"}을 이용해 능동적으로 중요 기억을 선별함. 특히, 학습 종료 후에도 추론 단계에서 실시간으로 학습하는 \textbf{Test-Time Training (TTT)}을 수행하여 급변하는 생체 리듬과 환경에 즉각 적응함.
\item   \textbf{Global Neural Workspace (GNW)}: Dehaene 등[5]의 GNW 이론에 따라, Titans는 감각/뇌 정보가 경쟁·통합되는 "의식적 작업 공간" 역할을 수행함.
\item   \textbf{Graph-Informed Reasoning}: Moontae Lee(LG AI Research) 심층 추론 방법론\cite{lee2024strategic}을 적용, Titans 메모리에 저장된 에피소드들을 그래프 구조로 연결하여 단순 회상(Recall)을 넘어선 \textbf{전략적 추론(Strategic Reasoning)}을 수행함.
\item   \textbf{Tubular Manifold Alignment}: 통합 과정에서 모델은 Bertram 등[4]의 \textbf{"Tubularity(관형 구조)"}를 유지하도록 학습되어 강건성을 확보함.
\end{itemize}

\subsection{결론 및 차별점}
본 아키텍처는 단순 멀티모달 결합(Early Fusion)을 넘어, \textbf{LTC의 설명 가능한 인과성(Explainable Causality)}과 \textbf{Titans/SSM의 대규모 연산 효율성}을 결합한 \textbf{'Hybrid Liquid-SSM'} 아키텍처임. 이를 통해 IITP RFP가 요구하는 "인간 수준 인지 능력을 갖춘 뇌 내재화 모델"을 실현함.
