\section{제4장: 구현 및 검증 (Implementation & Validation)}

\subsection{4.1 구현 로드맵 (Micro to Macro)}
본 과제는 복잡한 뇌 기반 AI를 단계적으로 구현하기 위해 \textbf{3단계 마이크로-매크로 전략}을 수립함.

\subsubsection{4.1.1 Phase 1: Toy Model (마이크로 검증)}
\begin{itemize}
\item   \textbf{목표}: Dual Encoder와 Titans Memory의 핵심 동작 원리를 최소 단위에서 검증함.
\item   \textbf{데이터셋}:
\end{itemize}
    -   \textbf{Visual}: Moving MNIST (시간적 변화가 포함된 단순 시각 데이터).
    -   \textbf{Brain}: Synthetic EEG (LTC Network로 생성한 인공 뇌파).
\begin{itemize}
\item   \textbf{아키텍처}:
\end{itemize}
    -   Visual Encoder (Small CNN) + Brain Encoder (LTC) -> Concatenate.
    -   Titans Memory (Mamba 기반)가 시퀀스 기억.
\begin{itemize}
\item   \textbf{학생 과제}: 위 Toy Model을 구축하여 "Multi-modal Memory Surprise"가 발생하는지 확인하는 것이 최우선 과제임.
\end{itemize}

\subsubsection{4.1.2 Phase 2: Scale-up (매크로 확장)}
\begin{itemize}
\item   \textbf{데이터셋}: BCI Competition IV (실제 뇌파) + Ego4D (1인칭 비디오/신체 움직임).
\item   \textbf{아키텍처}: ViT-Base + Liquid-S4 (대형 모델) 적용.
\item   \textbf{학습}: Google Cloud / NVIDIA DGX A100 인프라 활용.
\end{itemize}

\subsection{4.2 검증 지표 (Validation Metrics)}
| 지표 | 정의 (Definition) | 목표치 (Target) |
| :--- | :--- | :--- |
| \textbf{Tubularity} | Neural Manifold의 원통형 구조 유지 정도 (Bertram et al., 2026). | Baseline 대비 +30% |
| \textbf{Memory Capacity} | Titans가 저장 가능한 유효 시퀀스 길이 (Effective Context Length). | > 100k tokens |
| \textbf{Allostatic Efficiency} | 작업 수행 시 소모되는 연산 에너지 효율 (Energy per Task). | 기존 Transformer 대비 50% 절감 (LTC 효과) |

\subsection{4.3 결론}
본 연구는 \textbf{신체성(Embodiment)}과 \textbf{뇌 동역학(LTC)}을 결합하여, 단순 AI를 넘어선 \textbf{"강건하고 효율적인 뇌 내재화 모델"}을 제시함. 이는 차세대 AI(Next-AI)의 표준이 될 것임.
