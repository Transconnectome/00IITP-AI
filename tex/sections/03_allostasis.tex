\section{제3장: 내수용 감각장 및 알로스태틱 뉴로트윈 (Allostatic Neuro-Twin)}

\subsection{3.1 개요: 예측적 생존 본능 (Predictive Allostasis)}
기존의 AI 에이전트나 웰니스 앱은 문제가 발생한 후 반응하는 \textbf{Homeostasis(항상성, Reactive)} 모델에 그쳤음. 본 과제는 시스템이 변화하는 미래 환경을 예측하고 사전에 에너지를 조절하는 \textbf{Allostasis(알로스태시스, Predictive Regulation)} 개념을 도입하여, 에이전트의 생존성을 극대화함.

\begin{figure}[h]
\centering
\includegraphics[width=1.0\textwidth]{allostasis_vs_homeostasis_diagram.png}
\caption{Interoceptive Allostasis vs Homeostasis}
\end{figure}

\subsection{3.2 시스템 구조: Neuro-Twin Loop}
사용자의 신체 및 환경 데이터를 실시간으로 미러링하는 \textbf{"Digital Neuro-Twin"}을 구축함.

\subsubsection{3.2.1 Ubiquitous & Visceral Sensing (입력)}
\begin{itemize}
\item   \textbf{Visceral Interoception}: 김성연 교수팀\cite{kim2020neural}의 연구에 기반하여, 위장 팽창(Gastric Distension)과 같은 내부 장기 신호를 \textbf{NTS $\rightarrow$ PVH $\rightarrow$ aIC} 경로를 통해 실시간으로 추적함. 이는 단순한 포만감을 넘어, 에이전트의 \textbf{기초 에너지 수준(Basal Energy Level)}을 정의하는 핵심 변수로 작용함.
\item   \textbf{Passive Sensing}: 스마트워치/글래스 등 웨어러블 기기를 통해 심박 변이도(HRV), 피부 전도도(EDA) 등 자율신경계 신호를 수동적으로 수집함.
\item   \textbf{Neural Proxy based on Gastric State}: 위장 상태(Gastric State)가 뇌의 각성(Arousal) 및 행동 모드(Exploration vs. Exploitation)를 조절한다는 이론\cite{berntson2021neural}을 적용하여, 내장 신호로부터 뇌의 거시적 상태를 역추론함.
\end{itemize}

\subsubsection{3.2.2 Energy Landscape Inference (처리)}
뇌 상태를 고차원 \textbf{"Energy Landscape(에너지 지형)"} 상의 좌표로 매핑함.
\begin{itemize}
\item   \textbf{Pathological Attractor}: 우울, 불안, 또는 시스템 과부하 상태를 '빠져나오기 힘든 깊은 골짜기(Attractor)'로 정의함.
\item   \textbf{Allostatic Load}: 현재 상태가 지속될 경우 시스템에 가해질 누적 부하(Load)를 예측 시뮬레이션함.
\end{itemize}

\subsubsection{3.2.3 Predictive Nudge (개입)}
시스템이나 사용자가 병리적 끌개(Attractor)로 진입하기 전, 선제적으로 개입함.
\begin{itemize}
\item   \textbf{Intervention}: 조명, 온도 조절 또는 사용자에게 알림(Nudge)을 제공하여, 에너지 지형 상의 '안전 지대'로 궤적을 수정함[Psychology Today 2025].
\end{itemize}

\subsection{3.3 기대 효과}
단순한 헬스케어를 넘어, 극한 환경(재난, 우주 등)에서 인간과 AI 시스템의 공생(Symbiosis)을 가능케 하는 \textbf{"Energy-Efficient Robust Intelligence"}를 구현함.
