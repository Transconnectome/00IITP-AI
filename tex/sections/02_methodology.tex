\section{제2장: 검증 및 방법론 (Validation & Methodology)}

\subsection{2.1 Tubularity & Manifold Alignment (검증 지표)}
본 연구는 AI 모델이 단순한 패턴 매칭을 넘어, 인간 뇌와 유사한 \textbf{"Neural Manifold(신경 다양체)"}를 형성하는지 검증하기 위해 Bertram et al.[4]이 제안한 혁신적 지표인 \textbf{"Tubularity"}를 도입함.

\subsubsection{2.1.1 Neural Manifold Alignment}
이종(Heterogeneous) 데이터(시각, 언어, 뇌파)가 통합될 때, 각 모달리티의 잠재 공간(Latent Space)이 서로 엉키지 않고 매끄럽게 정렬되어야 강건한 추론이 가능함.
\begin{itemize}
\item   \textbf{Tubularity Metric}: 학습된 궤적(Trajectory)이 국소적으로 원통형(Tubular) 구조를 유지하는지 측정함. 이는 모델이 노이즈에 얼마나 강건한지(Robustness)를 나타내는 핵심 지표임.
\item   \textbf{검증 목표}: DIVER-Neuro 모델의 Tubularity Score가 기존 Transformer 기반 모델 대비 30 percent 이상 향상됨을 입증함.
\end{itemize}

\section{실험 및 검증 방법론 (Methodology)}

\subsection{데이터셋 구축 (Dataset Construction)}
본 연구는 뇌 활동(Brain)과 외부 환경(Vision)을 동시에 포함하는 멀티모달 데이터셋을 활용함.

\subsubsection{Toy Dataset: Moving MNIST + Synthetic EEG}
\begin{itemize}
\item   \textbf{Visual}: $28 \times 28$ Moving MNIST (2 digits).
\item   \textbf{Brain (Synthetic)}: 비선형 Izhikevich 뉴런 모델을 사용하여, Visual Stimulus의 속도와 위치에 반응하는 가상의 EEG 생성.
\item   \textbf{목적}: Dual Encoder의 기본 동작 및 Latent Alignment 검증.
\end{itemize}

\subsubsection{Main Dataset: BCM-V (Brain-Computer Interface Multimodal Video)}
\begin{itemize}
\item   \textbf{구성}: 피험자가 1인칭 시점(Ego-centric) 영상을 시청할 때 측정된 128채널 EEG 및 Eye-tracking 데이터.
\item   \textbf{특징}: 자연스러운 환경에서의 시각-뇌 활동 상관관계를 포함.
\end{itemize}

\subsection{학습 전략 (Training Strategy)}

\subsubsection{Phase 1: Pre-training (Brain-Visual Alignment)}
\begin{itemize}
\item   \textbf{Contrastive Learning (CLIP Style)}: OmniField(Visual)와 Brain Encoder(LTC)의 Latent Vector $z_v, z_b$ 간의 코사인 유사도를 최대화함.
\item   \textbf{Masked Modeling (MAE Style)}: Brain Encoder의 일부 타임스텝을 마스킹하고 복원하도록 학습하여 시간적 문맥(Temporal Context) 학습.
\end{itemize}

\subsubsection{Phase 2: Titans Memory Integration}
\begin{itemize}
\item   \textbf{Surprise-based Update}: 정렬된 Latent Vector를 Titans에 입력. 예측 오차(Prediction Error)가 높은 "놀라운(Surprising)" 정보만 Long-term Memory에 저장.
\item   \textbf{Free Energy Minimization}: MIRAS의 \textbf{Surprise Metric}은 수학적으로 Friston의 \textbf{Free Energy(예측 불확실성)}와 동등함\cite{friston2010free}. Titans는 이 Free Energy를 최소화하는 방향으로 메모리를 갱신함으로써, 생물학적 항상성(Homeostasis) 원리를 구현함.
\end{itemize}

\subsubsection{Brain-Tuning: Semantic Alignment}
DIVER의 fMRI 데이터를 사용하여 Titans의 상위 레벨 표상을 의미론적으로 정렬합니다 \cite{benara2025braintuning}.

\subsubsection{Biological Validation: Gastric Interoception (Murine Data)}
Titans의 기억 메커니즘이 실제 생물학적 회로와 유사하게 작동하는지 검증하기 위해, \textbf{김성연 교수팀의 동물 실험 데이터}를 활용합니다 \cite{kim2020neural}.
*   \textbf{Data Source}: 생쥐(Mouse)의 \textbf{PB Pdyn 뉴런 Two-photon Calcium Imaging} 데이터 (위장 팽창 시의 단일 뉴런 활성).
*   \textbf{Hypothesis}: Titans의 \textbf{Surprise Metric}은 생쥐의 PB Pdyn 뉴런이 보여주는 \textbf{Sustained Activity (Integrator)} 패턴과 높은 상관관계를 보여야 합니다.
*   \textbf{Validation}: 위장 팽창(Gastric Distension) 시뮬레이션 환경에서, Titans의 Memory State 변화가 실제 신경 데이터의 역동성과 일치하는지 정량적으로 분석합니다. 이는 AI가 단순한 정보 저장이 아닌, 생체와 유사한 \textbf{'내부 상태 적분(Internal State Integration)'}을 수행함을 증명합니다.

\subsection{실험 및 검증 계획 (Validation Plan)}
\begin{itemize}
\item   \textbf{Phase 1}: Toy Dataset(Moving MNIST + Synthetic EEG)을 이용한 개념 증명.
\item   \textbf{Phase 2}: BCM-V(뇌-컴퓨터 인터페이스) 데이터셋을 활용한 대규모 학습 및 Tubularity 측정.
\end{itemize}
