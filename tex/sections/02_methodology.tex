\section{제2장: 검증 및 방법론 (Validation & Methodology)}

\subsection{2.1 Tubularity & Manifold Alignment (검증 지표)}
본 연구는 AI 모델이 단순한 패턴 매칭을 넘어, 인간 뇌와 유사한 \textbf{"Neural Manifold(신경 다양체)"}를 형성하는지 검증하기 위해 Bertram et al.[4]이 제안한 혁신적 지표인 \textbf{"Tubularity"}를 도입함.

\subsubsection{2.1.1 Neural Manifold Alignment}
이종(Heterogeneous) 데이터(시각, 언어, 뇌파)가 통합될 때, 각 모달리티의 잠재 공간(Latent Space)이 서로 엉키지 않고 매끄럽게 정렬되어야 강건한 추론이 가능함.
\begin{itemize}
\item   \textbf{Tubularity Metric}: 학습된 궤적(Trajectory)이 국소적으로 원통형(Tubular) 구조를 유지하는지 측정함. 이는 모델이 노이즈에 얼마나 강건한지(Robustness)를 나타내는 핵심 지표임.
\item   \textbf{검증 목표}: DIVER-Neuro 모델의 Tubularity Score가 기존 Transformer 기반 모델 대비 30% 이상 향상됨을 입증함.
\end{itemize}

\subsubsection{2.1.2 Untangling Fraction}
복잡하게 얽힌 뇌 신호 데이터를 모델이 얼마나 효과적으로 풀어서(Untangle) 선형적으로 분리 가능한 상태로 만드는지 평가함.
\begin{itemize}
\item   \textbf{DiCarlo 방법론 적용}: 영장류 시각 피질 연구(DiCarlo et al.)에서 사용된 Object Manifold Untangling 지표를 AI 모델 평가에 차용함.
\end{itemize}

\subsection{2.2 Titans Memory Mechanism (메모리 방법론)}
장기 의존성(Long-term Dependency) 문제 해결을 위해 Google의 최신 아키텍처인 \textbf{Titans}[1]를 본 과제에 맞게 최적화함.

\subsubsection{2.2.1 Surprise-based Memory Update}
모든 정보를 저장하는 것은 비효율적임. Titans의 핵심 원리인 \textbf{"Memory Surprise"}를 활용하여, 예측 오차(Prediction Error)가 큰 정보만을 선별적으로 장기 기억(Long-term Memory)에 저장함.
\begin{itemize}
\item   \textbf{효과}: 메모리 효율성을 극대화하고, 불필요한 배경 노이즈를 자동으로 필터링함.
\end{itemize}

\subsubsection{2.2.2 Dual Memory System}
\begin{itemize}
\item   \textbf{Short-term (Core)}: 현재 작업(Task) 수행을 위한 빠른 주의(Attention) 매커니즘.
\item   \textbf{Long-term (Neural Memory)}: 심층 신경망 가중치(Weights) 형태로 저장되는 암묵적 기억(Implicit Memory). SSM(State Space Model)을 통해 수천 스텝 이전의 정보도 손실 없이 인출함[3].
\end{itemize}

\subsection{2.3 Brain-Tuning: 생물학적 정렬 (Biological Alignment)}
Mariya Toneva 등\cite{benara2025braintuning}이 제안한 \textbf{"Brain-Tuning"} 기법을 도입하여, 인공신경망의 잠재 표현을 실제 인간 뇌 반응(fMRI/EEG)과 정렬(Alignment)시킴.
\begin{itemize}
\item   \textbf{Method}: Titans 모델의 중간 레이어 출력을 인간 뇌의 언어/감각 피질 반응 데이터(DIVER Dataset)와 매핑되도록 지도 학습(Supervised Fine-tuning)함.
\item   \textbf{Effect}: Toneva의 연구\cite{policzer2025multimodal}에서 입증되었듯, 단순 데이터 학습 대비 \textbf{의미론적 이해(Semantic Understanding)} 능력이 대폭 향상되며, 인간과 유사한 계층적 정보 처리(Hierarchical Processing) 구조를 갖추게 됨.
\end{itemize}

\subsection{2.4 실험 및 검증 계획 (Validation Plan)}
\begin{itemize}
\item   \textbf{Phase 1}: Toy Dataset(Moving MNIST + Synthetic EEG)을 이용한 개념 증명.
\item   \textbf{Phase 2}: BCM-V(뇌-컴퓨터 인터페이스) 데이터셋을 활용한 대규모 학습 및 Tubularity 측정.
\end{itemize}
